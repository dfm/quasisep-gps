\documentclass[10pt]{article}

\usepackage{amsmath}
\usepackage{amssymb}
\usepackage{graphicx}

% Change spacing in lists
\usepackage{enumitem}
\setlist{noitemsep}

\usepackage{natbib}
\bibliographystyle{aasjournal}
\setcitestyle{round, citesep={,}, aysep={}}

\definecolor{linkcolor}{rgb}{0.36, 0.54, 0.66}
\hypersetup{
  hidelinks,
  colorlinks=true,
  linkcolor=linkcolor,
  citecolor=darkgray,
  urlcolor=linkcolor}

% My macros
\newcommand{\project}[1]{\textsf{#1}}
\newcommand{\mv}[1]{\ensuremath{\boldsymbol{#1}}}
\newcommand{\mm}[1]{\ensuremath{\boldsymbol{#1}}}
\newcommand{\T}{\ensuremath{^\mathrm{T}}}

% Journals
\newcommand\aj{{AJ}}
\newcommand\apj{{ApJ}}
\newcommand\apjl{{ApJL}}
\newcommand\apjs{{ApJS}}
\newcommand\mnras{{MNRAS}}
\newcommand\pasp{{PASP}}
\newcommand\aap{{A\&A}}

% Page layout stuff
\setlength{\topmargin}{0.0in}
\setlength{\headheight}{0.0in}
\setlength{\headsep}{0.0in}
\setlength{\parindent}{\baselineskip}
\setlength{\textwidth}{4.3in}
\setlength{\textheight}{2\textwidth}
\raggedbottom\sloppy\sloppypar\frenchspacing

\setlength{\marginparsep}{0.15in}
\setlength{\marginparwidth}{2.7in}
\usepackage{marginfix} 
\newcounter{marginnote}
\setcounter{marginnote}{0}
\renewcommand{\footnote}[1]{\refstepcounter{marginnote}\textsuperscript{\themarginnote}\marginpar{\color{darkgray}\raggedright\footnotesize\textsuperscript{\themarginnote}#1}}
\newcommand{\tfigurerule}{\rule{0pt}{1ex}\\ \rule{\marginparwidth}{0.5pt}\\ \rule{0pt}{0.25ex}}
\newcommand{\bfigurerule}{\rule{0pt}{0.25ex}\\ \rule{\marginparwidth}{0.5pt}\\ \rule{0pt}{1ex}}
\renewcommand{\caption}[1]{\parbox{\marginparwidth}{\footnotesize\refstepcounter{figure}\textbf{\figurename~\thefigure}: {#1}}}

\setlength{\oddsidemargin}{0.5\paperwidth}
\addtolength{\oddsidemargin}{-1.0in}
\addtolength{\oddsidemargin}{-0.5\textwidth}
\addtolength{\oddsidemargin}{-0.5\marginparwidth}
\addtolength{\oddsidemargin}{-0.5\marginparsep}

\frenchspacing\sloppy\sloppypar\raggedbottom

\begin{document}

\section*{Scalable Gaussian Processes \\ with Quasiseparable Matrices\footnote{%
    This manuscript was prepared using the \href{https://github.com/rodluger/showyourwork}{\showyourwork} workflow \citep{Luger2021}. The source code used to generate this version can be found at \href{\GitHubURL/tree/\GitHubSHA}{\project{dfm/quasisep-gps}} on \project{GitHub}.
  }}

\noindent\textbf{Daniel Foreman-Mackey}\footnote{%
  The authors would like to thank the Astronomical Data Group at Flatiron for listening to every iteration of this project and for providing great feedback every step of the way.
  We would also like to thank
  Suzanne Aigrain (Oxford),
  Gregory Guida (Etsy),
  Christopher Stumm (Etsy),
  \emph{and others\ldots}
  for insightful conversations and other contributions.
}\\
{\footnotesize%
\textsl{Center for Computational Astrophysics, Flatiron Institute, New York, NY, USA}%
}

\medskip\noindent\textbf{Final author list and order TBD}\\
{\footnotesize%
\textsl{Affiliations}\\
\textsl{More affiliations}%
}

\paragraph{Abstract:}

An abstract. \citep{Foreman-Mackey2017, Foreman-Mackey2018}

\section{Quasiseparable matrices}

There exist a range of definitions for \emph{quasiseparable matrices} in the literature, so to be explicit, let's select the one that we will consider in all that follows.
The most suitable definition for our purposes is nearly identical to the one used by \citet{Eidelman1999}, with some small modifications that will become clear as we go.

Let's start by considering an $N \times N$ \emph{square quasiseparable matrix} $\mm{M}$ with lower quasiseparable order $m_l$ and upper quasiseparable order $m_u$.
In this note, we represent this matrix $\mm{M}$ as:\footnote{Comparing this definition to the one from \citet{Eidelman1999}, you may notice that we have swapped the labels of $\mv{g}_j$ and $\mv{h}_i$, and that we've added an explicit transpose to $\mm{B}_k\T$. These changes simplify the notation and implementation for symmetric matrices where, with our definition, $\mv{g} = \mv{p}$, $\mv{h} = \mv{q}$, and $\mm{B} = \mm{A}$.}
%
\begin{eqnarray}\label{eq:square-qsm-def}
  M_{ij} &=& \left \{ \begin{array}{ll}
    d_i\quad,                                                                   & \mbox{if }\, i = j \\
    \mv{p}_i\T\,\left ( \prod_{k=i-1}^{j+1} \mm{A}_k \right )\,\mv{q}_j\quad,   & \mbox{if }\, i > j \\
    \mv{h}_i\T\,\left ( \prod_{k=i+1}^{j-1} \mm{B}_k\T \right )\,\mv{g}_j\quad, & \mbox{if }\, i < j \\
  \end{array}\right .
\end{eqnarray}
%
where
%
\begin{itemize}
  \item $i$ and $j$ both range from $1$ to $N$,
  \item $d_i$ is a scalar,
  \item $\mv{p}_i$ and $\mv{q}_j$ are both vectors with $m_l$ elements,
  \item $\mm{A}_k$ is an $m_l \times m_l$ matrix,
  \item $\mv{g}_j$ and $\mv{h}_i$ are both vectors with $m_u$ elements, and
  \item $\mm{B}_k$ is an $m_u \times m_u$ matrix.
\end{itemize}
%
In \autoref{eq:square-qsm-def}, the product notation is a little sloppy so, to be more explicit, this is how the products expand:
%
\begin{eqnarray}
  \prod_{k=i-1}^{j+1} \mm{A}_k &\equiv& \mm{A}_{i-1}\,\mm{A}_{i-2}\cdots\mm{A}_{j+2}\,\mm{A}_{j+1}\quad\mbox{, and} \nonumber\\
  \prod_{k=i+1}^{j-1} \mm{B}_k\T &\equiv& \mm{B}_{i+1}\T\,\mm{B}_{i+2}\T\cdots\mm{B}_{j-2}\T\,\mm{B}_{j-1}\T \quad.
\end{eqnarray}

\begin{eqnarray}
  \mm{M} &=& \left(\begin{array}{cccc}
      d_1                                & \mv{h}_1\T\mv{g}_2         & \mv{h}_1\T\mm{B}_2\T\mv{g}_3 & \mv{h}_1\T\mm{B}_2\T\mm{B}_3\T\mv{g}_4 \\
      \mv{p}_2\T\mv{q}_1                 & d_2                        & \mv{h}_2\T\mv{g}_3           & \mv{h}_2\T\mm{B}_3\T\mv{g}_4           \\
      \mv{p}_3\T\mm{A}_2\mv{q}_1         & \mv{p}_3\T \mv{q}_2        & d_3                          & \mv{h}_3\T\mv{g}_4                     \\
      \mv{p}_4\T\mm{A}_3\mm{A}_2\mv{q}_1 & \mv{p}_4\T\mm{A}_3\mv{q}_2 & \mv{p}_4\T\mv{q}_3           & d_4                                    \\
    \end{array}\right)
\end{eqnarray}

\marginpar{\tfigurerule\\
  \includegraphics[width=\marginparwidth]{figures/mandelbrot.pdf}
  \caption{Face.\label{fig:mandelbrot}}\\
  \bfigurerule}

\clearpage\raggedright
\bibliography{bib}

\end{document}
